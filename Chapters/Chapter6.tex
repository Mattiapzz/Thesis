\chapter{Optimal Control Initialization}
\label{Ch:OCIni}
%
The problem of initialization of optimal control is of great importance for the convergence or even the existence of a solution of the problem. However, for some unknown reasons, the academic literature lacks of a deep investigation and discussion on this topic. Most of the time is given for granted or even ignore. In this chapter, the authors tries to give an overview on how the control problem can be initialised and the role played by each therm. Most of the time, the author, will refer to the equation \ref{eq:GenOCP}. Moreover, in the next section, a general idea and definition of continuation is given.
%
\section{Torque Control}
%
\begin{figure}[ht!]
    \centering
    %\includegraphics[width=\linewidth]{}
    \caption{}
    \label{fig:TorqueControl}
\end{figure}
%
\section{Lagrange term}
%
The lagrange term of the cost function $\mathcal{J}$ in equation \ref{eq:GenOCP} address all time varying parts of the optimal control problem. In the following section possible argument of the lagrange term are explained and explored.
%
\subsection{Minimum time}
%
In the previous chapter \ref{Ch:OCIni} the time-space change of coordinate is explained yielding equation \ref{eq:zetadot}.
%
\begin{equation}
    \label{eq:zetadot}
    \dot{s} = \frac{u(t) \cos(\xi(t)) - v(t) \sin(\xi(t))}{1-\kappa(s(t)) n(t)}
\end{equation}
%
with the lagrange integral part as 
%
\begin{equation}
    \int_{\zeta_L}^{\zeta_R} \mathcal{L}(\zeta)\mathrm{d}\zeta = \int_{\zeta_L}^{\zeta_R} \frac{1}{\dot{s}} \mathrm{d}\zeta  
\end{equation}
%
In this case one can introduce a scalar $w_t$ that weights the influence of the minimum time on the OCP yielding
%
\begin{equation}
    \int_{\zeta_L}^{\zeta_R} \mathcal{L}(\zeta)\mathrm{d}\zeta = \int_{\zeta_L}^{\zeta_R} \frac{w_t}{\dot{s}} \mathrm{d}\zeta  
\end{equation}
%
This could seem strange, but it is a useful expedient to yield a fast convergence in continuation. This will be explained later in this chapter a section \ref{sec:Continuation} and \ref{sec:Homotopy}.
%
\subsection{Penalty on unwanted dynamics}
%
In some optimal control problems unwanted dynamics or effect arises. Those could be not feasible or not physically consistent with what can happen in reality. For instance, even in extreme manoeuvres a real rider cannot and will not apply to fast change of couple or slip. This means that one can penalize this effect by penalising the square distance from a desired dynamic.\\
Lets follow the example of the slip. As suggested by Leonelli and Limebeer \cite{leonelli2019optimal} one can penalise the dynamic of the steering and the slip ($\dot{\lambda}_f$, $\dot{\lambda}_r$ and $\tau$) having
%
\begin{equation}
    \int_{\zeta_L}^{\zeta_R} \mathcal{L}(\zeta)\mathrm{d}\zeta = \int_{\zeta_L}^{\zeta_R} ( w_t + w_\delta \dot{\tau} + w_{\lambda_f} \dot{\lambda_f} + w_{\lambda_r} \dot{\lambda_r} ) \frac{1}{\dot{s}} \mathrm{d}\zeta  
\end{equation}
%
The weights $w_\delta$, $w_{\lambda_f}$ and $w_{\lambda_f}$ should be chosen carefully to both take care of the order of magnitude of the associated dynamic and do not influence the OCP solution.
%
\subsection{Steady state tacking}
%
In the lagrange term other kind of minimisation effect can be introduced. For instance tracking a steady state. One can think of using the entire steady state condition or part of it to minimise the square difference from this reference. Lets define a vector of state values in steady state as $\mathbf{x_{ss}}$. Therefore the mayer term became
%
\begin{equation}
    \int_{\zeta_L}^{\zeta_R} \mathcal{L}(\zeta)\mathrm{d}\zeta = \int_{\zeta_L}^{\zeta_R} [ w_t + (\mathbf{x}(\zeta)-\mathbf{x_{ss}})^T \mathbf{W_{ss}}  (\mathbf{x}(\zeta)-\mathbf{x_{ss}})  ] \frac{1}{\dot{s}} \mathrm{d}\zeta  
\end{equation}
%
where $\mathbf{W_{ss}}$ is a weighting matrix with the purpose of normalizing all terms.\\
Here the steady state is tracked, however any other reference state can be followed.\\
This technique is particularly useful when dealing with problem that are hard, non linear, slow converging or not converging at all. In fact, simple but rather smart idea is to limit the dynamic, internal and external, to yield a smoother solution with few or absent oscillation. This solution is easier to find because it is more stable.\\
This particular expedient will be investigate further in next sections.
%
\section{Continuation (Homotopy)} 
\label{sec:Continuation}   
%
In Optimal Control Problem the initialization of state, control and parameters play a major role. In fact, most of the time the convergence of the solution is highly dependent on boundary conditions and lagrange multipliers. Moreover, a convergence strategy working on a particular case could be ineffective in another set up.\\
A way to avoid and or mitigate this effect is to use continuation that in literature is also known as homotopy.\cite{effati2011solving}\\
The continuation is a simple, but rather smart expedient implemented in PINS and other software like in IPOPT. In the last one, the continuation can be performed only on parameters, while in PINS it can be performed also on penalties, weights and tolerances.\\
The general idea is to use weights and parameters inside the cost function to initially solve a problem slightly different from the minimum time. This simpler problem, usually, is fast converging and yield a solution that can be used as guess in the next problem (next step of the continuation). This gives much more information than the one imposed at the beginning. In particular, after the first convergence, the solution yields states, controls but also values of the lagrange multiplier. Those are initialized as null at the first iteration because unknown.\\
This first convergence solution can be used as guess solution in the next problem in which some parameters can be changed slowly pushing the problem to the minimum time one.\\
The idea is that once a solution is found, the solution of the next problem is just a perturbation of the previous. Hence, the initial complex OCP is transformed in a series of fast converging OCPs.\cite{effati2011solving}\\

%
\section{U-shape corner}
%
\begin{figure}[ht!]
    \centering
    %\includegraphics[width=\linewidth]{}
    \caption{}
    \label{fig:U-Shape}
\end{figure}
%
\section{S-shape corner}
%
\begin{figure}[ht!]
    \centering
    %\includegraphics[width=\linewidth]{}
    \caption{}
    \label{fig:S-Shapel}
\end{figure}
%
\input{Chapters/Chapter6_6.tex}
%