\section{Optimal Control with Motorcycle Dynamics}
\label{chapter1_2}
%
\subsection{Past works on the topic}
%
Several optimal control problem for vehicle dynamics have been proposed in the literature, starting from simple dynamics model increasing to complicated ones and with different applications. Some concern safe manoeuvre minimising a certain functional while others minimize the time to complete a manoeuvre or a circuit (minimum time OC problems). Dal Bianco \textit{et al.}\cite{dal2019comparison} offers a deep and detailed analysis regarding the state-of-the-art in optimal control problems for minimum time application.\\
In general, optimal control problems can be addressed in multiple ways. One method is the quasi-steady-state approach. In this case, the race line is precomputed and it is an input of the problem. The optimal control is therefore computed keeping the vehicle on the surface of a precomputed maximum performance envelope (G-G surface). The QSS (Quasi-Steady-State) approaches neglect all transient dynamics except for the longitudinal. The lateral acceleration is computed using the curvature of the trajectory and the tangential speed while the vehicle can only accelerate or decelerate in the longitudinal direction. This approach has the advantage of small computational times but it lacks accuracy due to the neglected transient dynamic.\\
Another method to approach the problem is the so-called transient optimal control. As the name suggests, the dynamic in no more neglected and the differential equations are the constraints of the minimization problem. This formulation allows to describe the full motion of the motorcycle and optimise both controls and trajectory. As addressed in the previous section, for complicated non-linear models the solution techniques are mainly three: differential dynamic programming (DDP), indirect optimal control (IOC) and direct optimal control (DOC).\\
However, for some reason, the academic research on optimal control is being focused on cars and four-wheeled vehicles leaving motorcycle almost uncovered. This is probably due to the high complexity of the model and the instability.\\
One of the first publication in the field of optimal control for motorcycle dynamics is from Sharp \cite{sharp2014method}. In his work, a dynamic model with 5 degrees of freedom is proposed assuming rolling without slipping and saturating the possible applied force. Other publications propose more complicated model considering also driver motion and gearbox effect\cite{cossalter2010investigation,lot2008advanced,simon2008application}. However, those do not investigate the effect of tyre forces using any tire model. Cossalter \textit{et al.}\cite{cossalter2010investigation} propose a model where the forces are bounded with an adherence ellipse.\\
Bertolazzi \textit{et al.}\cite{bertolazzi2005symbolic} presents the results for an optimal control problem with a simple model of a motorcycle obtained with the multi-body approach and it is one of the few OCP solved with indirect methods in literature.\cite{biral2016notes}\\
%
\subsection{Recent developments}
%
Cossalter \textit{et al.} propose a further complicated model with the purpose of safety manoeuvre and investigation of rider position effect.\cite{cossalter2013optimization,massaro2010virtual}.\\
It is worth to highlight the recent work of Leonelli and Limebeerb\cite{leonelli2019optimal}. In their publication, they solve an optimal control problem for minimum lap time of a racing motorcycle using a model with 7 degrees of freedom. As well as other they discard the dynamic effect due to suspension motion. However, their work stands out for the use of the Magic Formula \cite{pacejka2012tire} in the modelling of exchanged forces. Anyhow, one could point out that it is not clear which formulation they use or if they consider combined effect or pure longitudinal and lateral condition. Moreover, it seem like they do not investigate the effect of self aligning torque imposing only an overturning moment to counterbalance the difference between the real point of application of forces and the theoretical one.\\
Another advance in their research is the use of three-dimensional curvilinear coordinate in combination with a complex model. This publication solves the OCP with the direct method using GPOPS-II\cite{patterson2014gpops}, which is a direct pseudo-spectral method based on Legendre–Gauss–Radau collocation and Radau’s integration formula.\\
It is important to highlight that they chose as control variable the steering torque and the slips. Many other publications follow this approach, even if it is physically unrealistic. 
%
\section{Unexplored problems and thesis goals}
%
Despite the recent developing in both motorcycle dynamics and optimal controls, there are several unexplored areas of research in this specific topic.\\
As far as the author know, in literature, no one successfully attempt to solve a minimum time optimal control problem with a complete motorcycle dynamic model controlled with torques. In fact, controlling the torque is a much harder problem that controlling in slips. This is due to the fast dynamic of the torque that causes a rapid change in the slip. This makes the problem unstable and special precautions are needed.\\
Moreover,in the academic literature, no one uses a "smart" approach to reduce the computational burden in optimal control. As far as the author know, in all motorcycle optimal control publications, the model is derived through the lagrange approach without taking into account that with the increasing number of DoF the differential equation became huge.\\
To summarise, given all the previous lack in scientific literature, the works of this thesis aim to:
%
\begin{itemize}
    \setlength{\itemsep}{0pt}
    \item Develop a dynamic model 
    \begin{itemize}
        \setlength{\itemsep}{0pt}
        \item that takes into account the internal motion of the suspensions
        \item written following a "smart" approach
    \end{itemize}
    \item solve a minimum time OCP controlling the slips
    \item solve a minimum time OCP controlling the torques
    \item investigate the effect and differences of the two models
\end{itemize} 
%