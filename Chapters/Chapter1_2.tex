\section{Optimal Control with Motorcycle Dynamics}
\label{chapter1_2}
%
Several optimal control problem for vehicle dynamics have been proposed in literature, starting from simple dynamics model to complicated ones and with different applications. Some concern safe manoeuvre minimising a certain functional while other minimize the time to complete a manoeuvre or a circuit (minimum time OC problems). Dal Bianco \textit{et al.}\cite{dal2019comparison} offers a deep and detailed analysis regarding the state-of-the-art in optimal control problems for minimum time application.\\
In general optimal control problems can be addressed in multiple ways. One method is the quasi steady state approach. In this case the race line is precomputed and it is an input of the problem. The optimal control is therefore computed keeping the vehicle on the surface of a precomputed maximum performance envelope (G-G surface). The QSS (Quasi Steady State) approach neglect all transient dynamics with the exception of the longitudinal. The lateral acceleration is computed using the curvature of the trajectory and the tangential speed while the vehicle can only accelerate or decelerate in the longitudinal direction. This approach have the advantage of small computational times but it lacks of accuracy due to the neglected transient dynamic.\\
Another method to approach the problem is the so called transient optimal control. As the name suggest, the dynamic in no more neglected and the differential equation are the constraint of the minimization problem. This formulation allow to describe the full motion of the motorcycle and optimise both controls and trajectory. As addressed in the previous section, for complicated non-linear models the solution techniques are mainly three: differential dynamic programming (DDP), indirect optimal control (IOC) and direct optimal control (DOC).\\
However, for some reason, the academic research on optimal control is being focused on cars and four wheeled vehicles leaving motorcycle almost uncovered. This is probably due to the high complexity of the model and the instability.\\
One of the fist publication in the field of optimal control for motorcycle dynamics is from Sharp \cite{sharp2014method}. In his work he proposes a dynamic model with 5 degrees of freedom assuming rolling without slipping and saturating the possible applied force. Other publications propose more complicated model considering also driver motion and gearbox effect\cite{cossalter2010investigation,lot2008advanced,simon2008application}. However, those do not investigate the effect of tyre forces using any tire model. Cossalter \textit{et al.}\cite{cossalter2010investigation} propose a model where the forces are bounded with an adherence ellipse.\\
Bertolazzi \textit{et al.}\cite{bertolazzi2005symbolic} presents the results for an optimal control problem with a simple model of a motorcycle obtained with the multi-body approach and it is one of the few OCP solved with indirect methods in literature.\cite{biral2016notes}\\
Cossalter \textit{et al.} propose a further complicated models with the purpose of safety manoeuvre and investigation of rider position effect.\cite{cossalter2013optimization,massaro2010virtual}.
It is worth to highlight the recent work of Leonelli and Limebeerb\cite{leonelli2019optimal}. In their publication they solve an optimal control problem for minimum lap time of a racing motorcycle using a 7 degrees of freedom model. As well as other they discard the dynamic effect due to suspension motion. However, their work stand out for the use of the Magic Formula \cite{pacejka2012tire} in the modelling of exchanged forces. Another advance in their research is the use of three-dimensional curvilinear coordinate in combination with a complex model. This publication solve the OCP with the direct method using GPOPS-II\cite{patterson2014gpops}, which is a direct pseudo-spectral method based on Legendre–Gauss–Radau collocation and Radau’s integration formula.
%