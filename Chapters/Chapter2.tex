\chapter{Motorcycle model} 
\label{Ch:MotorcycleModel}
%
There are multiple way to derive describe the behaviour of the motorcycle and to derive the equation of motion. Some use the lagrangian approach to use a minimum set of coordinates\cite{pacejka2006tyre,sharp2004advances,leonelli2019optimal}. Other derive the equation of motion using Newton-Euler equations.\\
In this thesis the dynamic model of the motorcycle has been derived using a multi-body approach and assuming the ISO convention for the orientation of the $z$-axis (upward). To this end multiple reference frames, points, bodies, forces and torques are defined in the following sections. The model was derived symbolically using Maple and MBSymba that deals with rotation and transformation matrices using homogeneous coordinates. In particular, working on the model, the author chose to derive the kinematics using a combination of global and recursive approach in order to use a minimum set of coordinates while containing the size of the derived equations. Moreover, in this work the author derive the equation without using the lagrangian approach, writing the equation of motion with the Newton-Euler approach.\\
In addition, the multi-body model is build with a "smart" approach that take advantages of dummy bodies to reduce the size of the equations. This particular techniques ie explained in section \ref{subsec:Dummy}. All the data and parameters of the motorcycle are reported in appendix \ref{app:MotoData}.
% 
\section{Motorcycle model with fixed suspension controlling slip}

There are multiple way to derive describe the behaviour of the motorcycle and to derive the equation of motion. Some use the lagrangian approach to use a minimum set of coordinates\cite{pacejka2006tyre}\cite{sharp2004advances}\cite{leonelli2019optimal}. Other derive the equation of motion using Newton-Euler equations.\\
In this thesis the dynamic model of the motorcycle has been derived using a multi-body approach and assuming the ISO convention for the orientation of the $z$-axis (upward). To this end multiple reference frames, points, bodies, forces and torques are defined in the following sections. The model was derived symbolically using Maple and MBSymba that deals with rotation and transformation matrices using homogeneous coordinates. In particular, working on the model, the author chose to derive the kinematics using a combination of global and recursive approach in order to use a minimum set of coordinates while containing the size of the derived equations. 

\section{Reference frames}

A common choice is to start by defining a reference frame in movement with respect to the ground, or fixed, one. This reference frame have in general three linear and three angular velocities. However, for the purpose of this thesis, we consider only planar roads. This means that we need to define the movement of a frame in a plane, therefore only three degrees of freedom are needed (three velocities). The moving reference frame ia addressed as $RF_1$ and has velocities $u(t)$, $v(t)$ and $\Omega(t)$. This set of velocities, also called quasi-coordinates, are suitable to be used later in the definition of curvilinear coordinate. $RF_1$ has the $x$-axis aligned with the direction of motion of the motorcycle. \\
The frame $RF_\phi$ is the reference frame attached to a plane rotated of an angle $\phi(t)$ commonly addressed as rolling angle around the moving $x$-axis and it is obtained recursively multiplying $RF_1$ for a rotation matrix.
\begin{equation}
    RF_\phi = RF_1 
    \left[ \begin {array}{cccc} 1&0&0&0\\ \noalign{\medskip}0&\cos
    \left( \phi \left( t \right)  \right) &-\sin \left( \phi \left( t
    \right)  \right) &0\\ \noalign{\medskip}0&\sin \left( \phi \left( t
    \right)  \right) &\cos \left( \phi \left( t \right)  \right) &0
   \\ \noalign{\medskip}0&0&0&1\end {array} \right]   
\end{equation}

Then a reference frame attached to the joint between the swingarm and the rear frame is defined with a translation in the vertical direction of the rolled frame of a certain height $h(t)$ and a rotation around the rolled $y$-axis of an angle $\theta(t)$ plus the caster angle $\epsilon$. The new frame will be from here addressed as $RF_{Rear}$
\begin{equation}
    RF_{Rear} = RF_\phi 
    \left[ \begin {array}{cccc} \cos \left( \theta \left( t \right) +
    \epsilon \right) &0&-\sin \left( \theta \left( t \right) +\epsilon
    \right) &0\\ \noalign{\medskip}0&1&0&0\\ \noalign{\medskip}\sin
    \left( \theta \left( t \right) +\epsilon \right) &0&\cos \left( 
    \theta \left( t \right) +\epsilon \right) &h \left( t \right) 
    \\ \noalign{\medskip}0&0&0&1\end {array} \right] 
\end{equation}
From $RF_{Rear}$ frame one can define the reference frame attached to the swingarm and the one attached to the steering assembly. The first is obtained with a rotation around the $y$-direction of the previous reference frame of a relative angle $eta(t)$ and a translation of the length of the swingarm. The advantage of choosing this relative angle is that there are already define relationship between this rotation an the force of the suspension. The new reference frame is addressed as $RF_\eta$ and will coincide with the centre of the rear wheel.
\begin{equation}
    RF_{\eta} = RF_{Rear} 
    \left[ \begin {array}{cccc} \cos \left( \eta \left( t \right) 
    \right) &0&\sin \left( \eta \left( t \right)  \right) &-\cos \left( 
    \eta \left( t \right)  \right) L_{{\it swa}}\\ \noalign{\medskip}0&1&0
    &0\\ \noalign{\medskip}-\sin \left( \eta \left( t \right)  \right) &0&
    \cos \left( \eta \left( t \right)  \right) &\sin \left( \eta \left( t
    \right)  \right) L_{{\it swa}}\\ \noalign{\medskip}0&0&0&1
    \end {array} \right]
\end{equation}
The second reference frame, the one attached to the steering assembly has already the $z$-axis with the same direction of the rotation axis of the steer. Therefore, $RF_\delta$ can be obtained with a translation of a fixed quantity in the $x$ direction and a rotation of an angle $\delta(t)$ around the rotation axis $z$. This steering angle is small for racing motorcycle and it is always smaller than $10$ degrees therefore can be linearised from here since the equation of motion are derived using Newton-Euler approach instead of Lagrange.
\begin{equation}
    RF_{\delta} = RF_{Rear} 
    \left[ \begin {array}{cccc} 1&-\delta \left( t \right) &0&L_{b}
    \\ \noalign{\medskip}\delta \left( t \right) &1&0&0
    \\ \noalign{\medskip}0&0&1&0\\ \noalign{\medskip}0&0&0&1\end {array}
     \right]    
\end{equation}
The reference frame attached to the centre of the front wheel is defined as $RF_{susp}$ and it is obtained with a translation in the negative vertical direction of a fixed quantity $s_{fs}$ plus a time-varying $s_f(t)$ which is the deformation of the suspension and a translation in the $x$ direction of $x_{off}$, an offset always present in the suspension fork.
\begin{equation}
    RF_{susp} = RF_\delta 
    \left[ \begin {array}{cccc} 1&0&0&x_{{\it off}}\\ \noalign{\medskip}0
    &1&0&0\\ \noalign{\medskip}0&0&1&-s_{{\it fs}}+s_{f} \left( t \right) 
    \\ \noalign{\medskip}0&0&0&1\end {array} \right] 
\end{equation}

In order to have a simplified model one can introduce two new reference frames one for the front and one for the rear wheel starting from $RF_1$. $RF_{FW}$ is defined with a translation of the components $x_f(t)$, $y_f(t)$ and $z_f(t)$ and a rotation of an angle $\delta_f(t)$ around the vertical direction and then one around the new longitudinal direction of angle $\phi_f(t)$. 

\section{Motorcycle model with fixed suspension controlling torque}
\section{Equations of motion}
%
The MBSymba library\cite{multibod60:online} allow to derive the newton euler equations of motion. As previously highlighted, the equation of motion needed are 11.
The first six are derived from the Newton and Euler equation of the whole system that is composed of all bodies, all anti-bodies and all forces at play. Th equation are projected in $RF_1$ and use the origin of $RF_1$ as a pole for the momentum equilibrium.\\ 
The equation are huge and it is pointless to show them if not in a simplified case. For instance with the internal degrees of freedom freezed and some other set to zero. Specifically the imposed values are:
%
\begin{equation}
    \label{eq:simplifyEQMotion}
    \phi(t) = 0, \eta(t)=\eta_00, \Omega(t) = 0, \delta(t) = 0, \theta(t) = \theta_00, s_f(t)=s_{f_{00}}
\end{equation}
%
The DoF set to zero are the roll angle of the motorcycle, the steering angle and the yaw rate. The motorcycle is in up-straight static condition. 
This yields a simplified version of the Newton equation such as:
%
\begin{equation}
\begin{array}{c}
\left( u \left( t \right)  \right) ^{2}{\it Ca}+M_{{\it tot}}\,{\frac {\rm d}{{\rm d}t}}u \left( t \right) -{\it Fxf} \left( t \right) -{\it Fxr} \left( t \right) = 0 \\
M_{{\it tot}}\,{\frac {\rm d}{{\rm d}t}}v \left( t \right) -{\it Fyf} \left( t \right) -{\it Fyr} \left( t \right) = 0\\
\displaystyle \left( m_{m}+m_{{\it rdr}}+m_{\delta}+m_{{\it swa}}+m_{{\it wf}}+m_{{\it wr}} \right) {\frac {{\rm d}^{2}}{{\rm d}{t}^{2}}}h \left( t \right) +M_{{\it tot}}\,g-{\it Fzf} \left( t \right) -{\it Fzr} \left( t \right) = 0
\end{array}
\end{equation}
%
The simplification performed transform the complex formulas of the dynamic of the motorcycle in something equal to the single track model of car.\\
The Euler equation are too complex to show even with such greater simplifications.\\
The equation of motion that should be derived from the steering dynamic is only one. It is the Euler equation around the $z$ axis of $RF_\delta$ and it is projected in this reference frame. As well as in the other cases the equation is long an complex. However the simplified version (with relationship in \ref{eq:simplifyEQMotion}) can be displayed.
%
\footnotesize
\begin{equation}
\displaystyle
\left( m_{{\it wf}}\,x_{{\it off}}+m_{\delta}\,x_{\delta} \right) {\frac {\rm d}{{\rm d}t}}v \left( t \right) + \left( {\it rf}\,\sin \left( \theta_{0}+\epsilon \right) -x_{{\it off}} \right) {\it Fyf} \left( t \right) -{\it Mzf} \left( t \right) \cos \left( \theta_{0}+\epsilon \right) +{\it Mxf} \left( t \right) \sin \left( \theta_{0}+\epsilon \right) -\tau \left( t \right)  = 0 
\end{equation}
\normalsize
%

The Euler equation of the rotation of the system composed by rear swingarm and rear wheel is derived using as pole the joint between rear frame and swingarm and it is projected in $RF_{Rear}$.
As for the previous the equation simplified is reported here.
%
\footnotesize
\begin{equation}
\begin{split}
\left( \left(  \left( L_{{\it swa}}-x_{{\it Swing}} \right) m_{{\it swa}}+L_{{\it swa}}\,m_{{\it wr}} \right) \cos \left( -\eta_{0}+\theta_{0}+\epsilon \right) +z_{{\it Swing}}\,m_{{\it swa}}\,\sin \left( -\eta_{0}+\theta_{0}+\epsilon \right) \right) {\frac {{\rm d}^{2}}{{\rm d}{t}^{2}}}h \left( t \right) \dots \\
\dots + \left( z_{{\it Swing}}\,m_{{\it swa}}\,\cos\left( -\eta_{0}+\theta_{0}+\epsilon \right) + \left( m_{{\it swa}}\,\left( -L_{{\it swa}}+x_{{\it Swing}} \right) -L_{{\it swa}}\,m_{{\it wr}} \right) \sin \left( -\eta_{0}+\theta_{0}+\epsilon \right) \right) {\frac {\rm d}{{\rm d}t}}u \left( t \right) + \dots \\
\dots + \left( -L_{{\it swa}}\,{\it Fzr} \left( t \right) - \left( m_{{\it swa}}\, \left( -L_{{\it swa}}+x_{{\it Swing}} \right) -L_{{\it swa}}\,m_{{\it wr}} \right) g \right) \cos \left( -\eta_{0}+\theta_{0}+\epsilon \right) + \dots\\
\dots + \left( z_{{\it Swing}}\,m_{{\it swa}}\,g+L_{{\it swa}}\,{\it Fxr} \left( t \right)  \right) \sin \left( -\eta_{0}+\theta_{0}+\epsilon \right) +\eta_{0}\,a_{1}\,k_{{\it rs}}+{\it Fxr} \left( t \right) {\it rr} +{\it Iy}_{{\it wr}}\,{\frac {{\rm d}^{2}}{{\rm d}{t}^{2}}}\theta_{r} \left( t \right)  = 0
\end{split}   
\end{equation}
\normalsize
%
The equation of motion describing the dynamic of front suspension is the $z$ component of the Newton equations of the system unsprung suspension plus front wheel. It is projected in $RF_\delta$
As for the previous the equation simplified is reported here.
%
\footnotesize
\begin{equation}
\cos \left( \theta_{0}+\epsilon \right) \left( {\frac {{\rm d}^{2}}{{\rm d}{t}^{2}}}h \left( t \right) \right) m_{{\it wf}}+ \left( gm_{{\it wf}}-{\it Fzf} \left( t\right)  \right) \cos \left( \theta_{0}+\epsilon \right) + \left( -\left( {\frac {\rm d}{{\rm d}t}}u \left( t \right)  \right) m_{{\it wf}}+{\it Fxf} \left( t \right)  \right) \sin \left( \theta_{0}+\epsilon \right) +s_{f_{0}}\,k_{{\it fs}} = 0
\end{equation}
\normalsize
%
The only two equation left out are the one describing the rotational dynamic of the rear and front wheels. Those are derived considering only the wheels and the forces applied on those. The Euler equations are projected respectively in $RF_{FW}$ and $RF_{RW}$. The simplified version for the front is the following
%
\begin{equation}
    {\it Iy}_{{\it wf}}\,{\frac {{\rm d}^{2}}{{\rm d}{t}^{2}}}\theta_{f}
    \left( t \right) +{\it Fxf} \left( t \right) {\it rf}-{\it Myf}
    \left( t \right)    
\end{equation}
%
While for the rear we have
%
\begin{equation}
    {\it Iy}_{{\it wr}}\,{\frac {{\rm d}^{2}}{{\rm d}{t}^{2}}}\theta_{r}
    \left( t \right) +{\it Fxr} \left( t \right) {\it rr}-{\it Myr}
    \left( t \right)        
\end{equation}
%

\section{Reduction to Single Track model}
%
As highlighted in the previous section, the complete equation of motion are complex and cannot be shown. However, one way to validate the model is to reduce the problem to simple case to observe the terms in the equation. As a proof of concept one can isolate the the two equation of motion of the whole motorcycle concerning the pitch and the vertical translation ($\theta$ and $h$).\\
%
%%% DISEGNO\\
%
From the equilibrium of momentum is clear that from those equation one can solve for the vertical forces. Those will have a complex formulation that simplified in the case of the motorcycle in vertical position ($\phi(t)=0$), no steering ($\delta(t)=0$), lateral velocity null ($v(t)=0$), zero yaw rate ($\Omega(t)=0$) and internal degrees of freedom freezed ($\eta(t)=\eta_{00}$, $\theta(t) = \theta_{00}$, $s_f(t)=s_{f_{00}}$). The equations are still complicated, but a lot of terms can be collected yielding the following expression.
%
\begin{equation}
    \label{eq:vertForces}
\begin{array}{c} 
\displaystyle
{\it Fzr} \left( t \right) =+{\frac {M_{{\it tot}}\,{\it ax} \left( t \right) {\it ZG}}{L}}+{\frac {{\it Ca}\,\left( u \left( t \right)  \right) ^{2}{\it ZA}}{L}}+{\frac {{\it Iy}_{{\it wf}}\,{\frac {{\rm d}^{2}}{{\rm d}{t}^{2}}}\theta_{f} \left( t \right) }{L}}+{\frac {{\it Iy}_{{\it wr}}\,{\frac {{\rm d}^{2}}{{\rm d}{t}^{2}}}\theta_{r} \left( t \right) }{L}}+{\frac {M_{{\it tot}}\,g{\it Lf}}{L}}\\
\displaystyle
{\it Fzf} \left( t \right) =-{\frac {M_{{\it tot}}\,{\it ax} \left( t \right) {\it ZG}}{L}}-{\frac {{\it Ca}\, \left( u \left( t \right)  \right) ^{2}{\it ZA}}{L}}-{\frac {{\it Iy}_{{\it wf}}\,{\frac {{\rm d}^{2}}{{\rm d}{t}^{2}}}\theta_{f} \left( t \right) }{L}}-{\frac {{\it Iy}_{{\it wr}}\,{\frac {{\rm d}^{2}}{{\rm d}{t}^{2}}}\theta_{r} \left( t \right) }{L}}+{\frac {M_{{\it tot}}\,g{\it Lr}}{L}}
\end{array}    
\end{equation}
%
In the previous equations $ax(t)$ is actually the longitudinal acceleration that in general is equal to $\frac {{\rm d}^{}}{{\rm d}{t}^{}}u(t)-\Omega(t) v(t)$. However, both $\omega(t)$ and $v(t)$ are considered null, therefore $\displaystyle ax(t) = \frac {{\rm d}^{}}{{\rm d}{t}^{}}u(t)$.
$L$ is the total length defined as $L= L_r+L_f$ where $L_r$ and $L_f$ are rear axis length and front axis length. Those are measure the distance of the contact point from the CoM of the motorcycle. $ZA$ is the height of the pressure point where the drag force is applied redefined in the reference frame $RF_1$.\\
The solution for the vertical forces in equation \ref{eq:vertForces} shows clearly the dependency on static load distribution between front and rear wheel due to the position of the centre of mass $\displaystyle{\frac {M_{{\it tot}}\,g{\it Lf}}{L}}$. The other therms depends on load transfer due to drag, to acceleration and wheels angular acceleration.\\
%
\begin{figure}[htb]
    \centering
    \includegraphics[width=\linewidth]{Coordinates/SingleTrackRed.pdf}
    \caption{Scheme of the single track reduction}
    \label{fig:SigngleTrackRed}
\end{figure}