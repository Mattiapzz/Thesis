\section{Reduction to Single Track model}
%
As highlighted in the previous section, the complete equation of motion are complex and cannot be shown. However, one way to validate the model is to reduce the problem to simple case to observe the terms in the equation. As a proof of concept one can isolate the the two equation of motion of the whole motorcycle concerning the pitch and the vertical translation ($\theta$ and $h$).\\
%
DISEGNO\\
%
From the equilibrium of momentum is clear that from those equation one can solve for the vertical forces. Those will have a complex formulation that simplified in the case of the motorcycle in vertical position ($\phi(t)=0$), no steering ($\delta(t)=0$), lateral velocity null ($v(t)=0$), zero yaw rate ($\Omega(t)=0$) and internal degrees of freedom freezed ($\eta(t)=\eta_00$, $\theta(t) = \theta_00$, $s_f(t)=s_{f_{00}}$). The equations are still complicated, but a lot of terms can be collected yielding the following expression.
%
\begin{equation}
    \label{eq:vertForces}
\begin{array}{c} 
\displaystyle
{\it Fzr} \left( t \right) =+{\frac {M_{{\it tot}}\,{\it ax} \left( t \right) {\it ZG}}{L}}+{\frac {{\it Ca}\,\left( u \left( t \right)  \right) ^{2}{\it ZA}}{L}}+{\frac {{\it Iy}_{{\it wf}}\,{\frac {{\rm d}^{2}}{{\rm d}{t}^{2}}}\theta_{f} \left( t \right) }{L}}+{\frac {{\it Iy}_{{\it wr}}\,{\frac {{\rm d}^{2}}{{\rm d}{t}^{2}}}\theta_{r} \left( t \right) }{L}}+{\frac {M_{{\it tot}}\,g{\it Lf}}{L}}\\
\displaystyle
{\it Fzf} \left( t \right) =-{\frac {M_{{\it tot}}\,{\it ax} \left( t \right) {\it ZG}}{L}}-{\frac {{\it Ca}\, \left( u \left( t \right)  \right) ^{2}{\it ZA}}{L}}-{\frac {{\it Iy}_{{\it wf}}\,{\frac {{\rm d}^{2}}{{\rm d}{t}^{2}}}\theta_{f} \left( t \right) }{L}}-{\frac {{\it Iy}_{{\it wr}}\,{\frac {{\rm d}^{2}}{{\rm d}{t}^{2}}}\theta_{r} \left( t \right) }{L}}+{\frac {M_{{\it tot}}\,g{\it Lr}}{L}}
\end{array}    
\end{equation}
%
In the previous equations $ax(t)$ is actually the longitudinal acceleration that in general is equal to $\frac {{\rm d}^{}}{{\rm d}{t}^{}}u(t)-\Omega(t) v(t)$. However, both $\omega(t)$ and $v(t)$ are considered null, therefore $\displaystyle ax(t) = \frac {{\rm d}^{}}{{\rm d}{t}^{}}u(t)$.
$L$ is the total length defined as $L= L_r+L_f$ where $L_r$ and $L_f$ are rear axis length and front axis length. Those are measure the distance of the contact point from the CoM of the motorcycle. $ZA$ is the height of the pressure point where the drag force is applied redefined in the reference frame $RF_1$.\\
The solution for the vertical forces in equation \ref{eq:vertForces} shows clearly the dependency on static load distribution between front and rear wheel due to the position of the centre of mass $\displaystyle{\frac {M_{{\it tot}}\,g{\it Lf}}{L}}$. The other therms depends on load transfer due to drag, to acceleration and wheels angular acceleration.\\
%
GRAFICI\\
%


