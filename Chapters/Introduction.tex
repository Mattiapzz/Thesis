\chapter*{Introduction}
\addcontentsline{toc}{chapter}{Introduction}
%
The simulation of mechanical and mechatronics systems allow to test and validate the design in a safe and efficient environment without the need to build the physical object and measure the parameters. The simulation is based on digital technology with major benefits as cost and efficiency and the possibility of an easy reconfiguration and retesting of a system which is usually impossible or infeasible for real model in terms of cost and time.\cite{maria1997introduction}\\
A large number of vehicle model are available in the scientific literature with different levels of complexity. Depending on the application a proper model must be chosen. Simple models are faster and therefore suitable for real-time purposes while complex models are time-consuming and are used in the case where the model cannot be simplified or the goal of the study is to replicate in detail the behaviour of the analysed system.\\
%
% MOTIVATION FOR THIS WORK
In literature a lot of studies concerning optimal control and in particular minimum lap time problems. However, most of them concern only four wheels vehicles. For some reasons, motorcycle manoeuvre had a minor interest in the research. This is due to the fact that the motorcycle model is highly non-linear and computational demanding to solve. Moreover, it is not always possible to reduce the system of dynamic equation in explicit form and most of the optimal control solver cannot deal with implicit forms.\\
Most of the optimal control problems for motorcycles in literature are solved using relatively simple dynamics and direct methods\cite{sharp2014method,leonelli2019optimal} or using the assumption of quasi-steady-state behaviour.\cite{cossalter1999general,simon2008application}\\
In some papers, the problem is solved using indirect approach\cite{bertolazzi2006symbolic}.\\ 
As far as the author knows the publication of Leonelli and Limber\cite{leonelli2019optimal} is the only one that takes into account tyre ground interaction while having a complicated dynamic model. However, it is not specified which version of the Magic formula is being used, moreover, the motorcycle suspensions are considered as fixed.\\   
%
The work of this thesis aims to derive different models of a racing motorcycle with increasing complexity taking into account the force exchange between tyre and ground using Pacejka's magic formula \cite{pacejka2006tyre}. The dynamic model has eleven degrees of freedom considering both moving and fixed suspensions and gyroscopic effect coming from wheels acceleration.\\ 
The models will then be used to calculate controls to achieve minimum time manoeuvre in different scenarios controlling the motorcycle with longitudinal slip or torque.\\
%
% WHICH ONE??
%
% The models will then be used to calculate controls and trajectory to achieve the minimum lap time of a specific circuit.\\
% In particular, the first model of the motorcycle will represent a vehicle with fixed suspensions meaning that the rear swingarm and the steering fork have respectively a fixed angle and a fixed length. The motorcycle is controlled with the steering torque and the longitudinal slip of front and rear wheel.\\
% The second model has again fixed suspensions, however, the vehicle is controlled with steering torque, braking torque at the front wheel and braking/traction torque at the rear.\\  
% The third model takes into account the internal motion due to suspension deformation and the motorcycle is again controlled with torques.\\
All the models are derived with the multi-body approach and symbolic formulation. To achieve this goal, the model is defined using Maple, a software well known for its capability in symbolic computation. Moreover, the equations of motion are obtained using MBSymba which is a custom library for Maple available online at \url{http://www.multibody.net}.\\
All three models are derived in a similar way as in the publication of Cossalter \textit{et al}\cite{cossalter2007influence}, without taking into account the lateral flexibility of the front torque and the torsional flexibility of the swingarm. The tyre forces are derived using the Pacejka's magic formula \cite{pacejka2006tyre} which is an empirical formula obtained from the assumption of similarity.\\
The minimum time trajectory and control is computed formulating a custom optimal control problem using XOptima package for Maple and then solving with PINS (the acronym for \textit{PINS Is Not a Solver}). The first is a library developed to transform the symbolic model of the vehicle (DAE), constraints, and target functions in C++ code that can be used by PINS. PINS is a software, free for academic purposes, developed at the University of Trento by Prof. Bertolazzi, Prof. Biral and Prof. Bosetti that can solve optimal control problems (OCPs) with the indirect method. As far as the author knows, there are not other optimal control solvers that exploit Pontryagin maximum principle and calculus of variations to solve the problem with the indirect method. \cite{bertolazzi2006symbolic}\\
The thesis is organized in the following way. 
In chapter \ref{Ch:StateArt}, there is a brief overview of the state of the art in optimal control techniques and motorcycle dynamic model for minimum time application.
Chapter \ref{Ch:MotorcycleModel} describe the derivation of kinematic and dynamic model of the motorcycle. 
In Chapter \ref{Ch:MagicFormula} tyre ground interaction is modelled following the magic formula of H. Pacejka\cite{pacejka2012tire} reporting formulas and tyre data used.
Chapter \ref{Ch:SS} report static condition solution and steady-state derived for the motorcycle. Those results are then used in chapter \ref{Ch:OCP} as a suitable initial condition to solve the optimal control problems.  
Chapter \ref{Ch:Results} presents and confront the results of the OCP.
Finally in the conclusions here is a wrap-up of all the obtained results with possible future developments.