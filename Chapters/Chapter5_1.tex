\section{General OCP}
%
In general the OCP problem is formulated in this way.
%
\begin{equation}
    \begin{aligned}
    &\underset{x(\cdot), u(\cdot)}{\operatorname{minimize}} \quad \mathcal{J}(x(t), u(t), t)\\
    &\begin{aligned}
        {\text { subject to }} \;
        & \mathcal{F}(x^{\prime}(t)x(t),u(t),t)=0, \quad t \in[a, b] \quad \text{(Dynamic System)}\\
        & b(x(a),x(b))=0, \quad \text{(Boundary Conditions)}\\
        & h(x(t),u(t),t)=0, \quad \text{(Equality constraints)}\\
        & g(x(t),u(t),t)>=0, \quad \text{(Inequality constraints)}\\
        & x(t) \in \mathbb{R}^{n_s}, \; u(t) \in \mathbb{R}^{n_c}
    \end{aligned}\\
    \end{aligned}
    \label{eq:GenOCP}
\end{equation}
%
where $\mathcal{J}x(\cdot)$ is the cost function that in general is written like the following.
%
\begin{equation}
    \mathcal{J}(x(t), u(t), t)=\mathcal{M}(x(a), x(b))+\int_{a}^{b} \mathcal{L}(x(t), u(t), t) \mathrm{d} t
\end{equation}
%
$\mathcal{F}(x^{\prime}(t)x(t),u(t),t)=0$ is the general expression for the system of ODE that can also be written in the form
%
\begin{equation}
    A(x(t), t) x^{\prime}(t)=f(x(t), u(t), t)
\end{equation}
%
in which all derivatives of the states appear linearly. Consequently, one can collect the mass matrix $A(x(t), t)$ which is not always invertible. The system of ODE is implicit in this case. This make the direct method hard to use if not impossible.\\
$x(t)$ is the vector of states of dimension $n_s$ and $u(t)$ is the vector of the controls of size $n_c$.
%
The aim of this study is to compute minimum time manoeuvre. Therefore the extremes of integration became $0$ and $T$ instead of $a$ and $b$. Moreover, the model constructed do not need any equality constraints. The lagrange function in this case is simple because of our interest in the minimum time problem.
%
\begin{equation}
    \mathcal{L}(x(t), u(t), t) = 1
\end{equation}
%
This combined with neglecting the Mayer therm $\mathcal{M}(\cdot)$ (most of the time in unnecessary) yields a cost function such as:
%
\begin{equation}
    \mathcal{J}(x(t), u(t), t) = \mathcal{M}(x(a), x(b))+\int_{0}^{T}  \mathrm{d} t = T 
\end{equation}
%
This lead to a simplified version of the problem
%
\begin{equation}
    \begin{aligned}
    &\underset{x(\cdot), u(\cdot)}{\operatorname{minimize}} \quad T \\
    &\begin{aligned}
        {\text { subject to }} \;
        & \mathcal{F}(x^{\prime}(t)x(t),u(t),t)=0, \quad t \in[0, T] \quad \text{(Dynamic System)}\\
        & b(x(0),x(T))=0, \quad \text{(Boundary Conditions)}\\
        & g(x(t),u(t),t)>=0, \quad \text{(Inequality constraints)}\\
        & x(t) \in \mathbb{R}^{n_s}, \; u(t) \in \mathbb{R}^{n_c}
    \end{aligned}\\
    \end{aligned}
    \label{eq:OCPminT}
\end{equation}
%
$T$ is a parameter and the problem cannot be solved in this therm. However there are several techniques to obtain a solvable OCP. The transformation will be discussed in the following sections.
%