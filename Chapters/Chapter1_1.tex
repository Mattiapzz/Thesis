\section{State of the art on Optimal Control Problems} 

Optimal control problem, also known as dynamic optimisation, are minimisation problem where the variables and parameters change with time. Dynamic systems are characterized by the states and often are controlled by a convenient choice of inputs (controls).\\
Dynamic optimisation aims to compute those controls and states for a dynamic system over a time interval to minimise one or more performance indexes. In other words, the input is chosen to optimize (minimize) an objective function while complying to constraint equations.

\subsection{Optimal Control Problems}
Optimal control problems are challenging from the theoretical point of view and of practical interest. However due to dimensionality and complexity of system of equations the application in real problems and industrial environment is still not so widespread.\\
In general, OPC can be continuous or discrete, linear or non-linear, time-variant or time-invariant. However, in this thesis are addressed only optimal control problems that are continuous time-variant and highly non-linear. Those properties will be discussed in the following sections. %[Ch.1.3]\\%reference real not this
In general, there are four main approaches to solve continuous-time OPC: state space approach, direct methods, indirect methods and differential dynamic programming.


\subsubsection{State-space approaches}

State-space approaches follow the principle of optimality for which each subarc of an optimal trajectory must be optimal. In literature, those are referred to as Hamilton-Jacobi-Bellman (HJB) equation. However, the problem needs numerical methods to be solved, moreover, a solution can be found only for small dimension problems due to \textit{course of dimensionality}. There is no practical application of this method to solve highly non-linear problem as a dynamic optimisation of a motorcycle model.
%citare fonti

\subsubsection{Direct Method}

Direct methods discretize the original optimal control problem into a  nonlinear programming problem (NLP). In other words, the OPC is transformed in a discrete-time system that can be solved using numerical schemes and  optimization techniques, namely Initial Value Solver (IVS) and Sequential Quadratic Programming (SQP) \cite{bertolazzi2005symbolic}
The main advantage of direct methods is the possibility to use inequality constraints even in case of change in the constraints active set ( activation/deactivation)\cite{biral2016notes}\\
Direct methods are easier to implement compared to the other three categories and this is one of the reasons why they are by far the most widespread. In fact, almost $90\%$ of the avaiable optimal control software rely on direct method. \cite{rao2009survey}\cite{rodrigues2014optimal}

\if FALSE
Brief descriptions
of three of the direct methods – single shooting, multiple shooting, and collocation
\fi

% allargare la descrizione
% cita molte + fonti
% 



\subsubsection{Indirect Method}

Indirect methods exploits the necessary condition of optimality to derive a boundary value problem (BVP) in ordinary differential equations(ODE). Therefore the BVP can be solved numerically as a non linear problem. The indirect method allow to first optimize and then discretize meaning that the problem con be firstly written in continuous time and discretized later using different discretization techniques.
The class of indirect methods exploits the well known calculus of variations
and the Euler-Lagrange differential equations, and the so-called Pontryagin
Maximum Principle.\cite{bertolazzi2006symbolic} \\
The numerical solution can be computed either by shooting techniques (single/multiple shooting) or by collocation.
The major drawbacks of indirect methods are that the problem could be difficult to solve or unstable due to the nature of the underlying differential equations (nonlinearity and instability) and the changes in control structure (active constraints in specific arcs). Moreover, in some arcs, singularity arises therefore the DAE index increase leading to the necessity of specialized solution techniques. \cite{biral2016notes}


% completa ampia descrizione dei metodi indiretti
% con esempi e citazioni di studi famosi
% cita testi e paper OCP e Bertolazzi+Biral
% Motivazioni per cui scegliere questo metodo
% 




\subsubsection{Differential Dynamic Programming}

% breve descrizione dei methodi


\subsection{Minimum time Optimal Control Problem}

% Problemi di minimo tempo
% Come sono stati risolti e con che methodi

\subsection{PINS}

% scegliere se mettrlo qui o nella parte con i metodi indiretti
% cita manuale e studio di Bertolazzi + Biral
% 

\if false
Some citation btw.
\cite{bertolazzi2005symbolic}
\cite{bertolazzi2006symbolic}
\cite{biral2016notes}
\cite{cossalter2006motorcycle}
\cite{dal2019comparison}
\cite{guiggiani2014science}
\cite{lot2014curvilinear,pacejka2006tyre}

\cite{pacejka2006tyre}
\cite{rao2009survey}
\cite{rodrigues2014optimal}
\cite{sharp2004advances}
\cite{sharp2014method}

\fi